\documentclass[a4paper, titlepage]{article}
\usepackage[sectionmark,fancysections]{polytechnique}
\usepackage{enumitem}
\usepackage{hyperref}
\usepackage{amsmath}
\usepackage{amssymb}
\usepackage[]{algorithm2e}

\renewcommand{\vec}[1]{\mathbf{#1}}
\renewcommand{\contentsname}{Sommaire}



\title{INF 552\\Image et vision par ordinateur}
\subtitle{Projet : Reconstruction d'images par apprentissage}
\date{17 Janvier 2016}
\author{Paul Michel\\Antoine Prouvost\\�l�ves en troisi�me ann�e}

\begin{document}
\maketitle

\tableofcontents

\cleardoublepage
\section*{Introduction}
\addcontentsline{toc}{section}{Introduction}

Nos motivations pour ce projet �taient de pouvoir implementer un algorithme d'apprentissage avec une vraie application au traitement des images.

Nous avons rapidement trouv� des travaux de recherche de l'Inria ainsi qu'un cours qui s'int�ressaient � ce sujet \cite{ref1}. L'id�e qui nus a tout de suite plut �tait de pouvoir reconstruire des images bruit�es, partiellement d�truites ou marqu�es.

Le fonctionnement de l'algorithme consiste � apprendre un dictionnaire de sous images, appel�s "patches", de l'image initiale ;  puis, d'utiliser celui-ci pour reconstruire chaque patche.

\cleardoublepage
\section{Algorithme et r�f�rences}
	\subsection{Dictionary Learning}
	Pour apprendre le dictionnaire, nous avons suivi les travaux de l'article \cite{ref3}. 
	
	Nous commen�ons par extraire l'ensemble des patches de l'image. Un patch est repr�sent� ici comme un vecteur $\vec{x} \in \mathbb{R}^{m}$ o� $m \in \mathbb{N}^{*}$ est la taille constante des patches (son nombre de pixels). Le dictionnaire, dont nous fixons la taille  $k \in \mathbb{N}^{*}$ est quant � lui repr�sent� par une matrice $\vec{D} \in \mathbb{R}^{m*k}$. Les �l�ments du dictionnaire sont stock�s en colones.\\
	
	\begin{algorithm}[H]
 		\KwData{$\lambda \in \mathbb{R}$, facteur de r�gularisation\\
		$T \in \mathbb{N}$, nombre d'it�ration de l'algorithme}
 		\KwResult{Un dictionnaire $\vec{D} \in \mathbb{R}^{m*k}$}
 		Remplir le dictionnaire initial $\vec{D}$ en choisissant al�atoirement des patches de l'image\;
		Initialiser deux matrices $\vec{A} \in \mathbb{R}^{m*m}$ et $\vec{B} \in \mathbb{R}^{m*k}$ � $0$\;
 		\For{$t=1$ to $T$}{
 		 	Tirer al�atoirement un patche $\vec{x} \in \mathbb{R}^{m}$ dans le dictionnaire (OU DANS LES PATCHS ?)\;
			Calculer :$$\alpha = \arg \min_{\alpha \in  \mathbb{R}^{m}} \frac{1}{2}\parallel \vec{x} - \vec{D}*\vec{\alpha}\parallel_{2}^{2} + \lambda\parallel \vec{\alpha} \parallel_{1} $$� l'aide de LARS\;
			$A = A + \alpha\alpha^{T}$\;
			$B = B + \vec{x}\alpha^{T}$\;
			\For{$j=1$ to $k$}{
				$\vec{u} = \frac{1}{\vec{A}_{j,j}}(\vec{b}_{j} - \vec{D}\vec{a}_{j})$\;
				$\vec{d}_{j} = \frac{1}{\max (\parallel\vec{u}\parallel_{2},1)}$\;
			}
 			 
	 	}
		 \Return D\;
 \caption{Online dictionary learning}
\end{algorithm}
	\subsection{Reconstruction d'image}

\cleardoublepage
\section{Implementation, r�sultats et interpr�tation}

\cleardoublepage
\bibliographystyle{plain}
\addcontentsline{toc}{section}{References}
\bibliography{biblio}
\end{document}
